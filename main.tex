\documentclass[12pt, oneside]{article}


%%%%%%%%%%%%%%%%%%%%%%%%%%%%
%%   Zusaetzliche Pakete  %%
%%%%%%%%%%%%%%%%%%%%%%%%%%%%
\usepackage[ngerman]{babel}
\usepackage{acronym}
\usepackage{enumerate}  
\usepackage{a4wide}
\usepackage{fancyhdr}
\usepackage{graphicx}
\usepackage{palatino}
\usepackage{float}
\usepackage{csquotes}
\usepackage[bookmarks]{hyperref}
\RequirePackage[ngerman=ngerman-x-latest]{hyphsubst}

\usepackage[style=authoryear-comp,dashed=false,natbib=true,backend=biber,maxnames=2]{biblatex}
\usepackage[justification=centering, labelfont=bf, textfont=bf]{caption}
\usepackage[a4paper, margin=0cm, left=3cm, top=2.5cm, right=2cm, bottom=3.5cm]{geometry} 
\usepackage{etoolbox}
\usepackage{acronym}
\usepackage{titlesec}
\usepackage[titles]{tocloft}
\usepackage{fontspec}

\setlength\cftparskip{-2pt}
\setlength\cftbeforesecskip{0pt}
\setlength\cftaftertoctitleskip{0pt}

\setmainfont{Times New Roman}

\DeclareCaptionFormat{myformat}{\fontsize{10}{0}\selectfont#1#2#3}
\captionsetup{format=myformat}

\setlength\parindent{0pt}
\setlength{\parskip}{6pt}
\setlength{\headheight}{0.8cm}

\linespread{1.25}

%\setlength{\cftbeforesecskip}{1.75pt}

\renewcommand{\cftfigpresnum}{Abbildung }
\renewcommand{\cfttabpresnum}{Tabelle }

\renewcommand{\cftfigaftersnum}{:}
\renewcommand{\cfttabaftersnum}{:}

\setlength{\cftfignumwidth}{2.75cm}
\setlength{\cfttabnumwidth}{2cm}

\setlength{\cftfigindent}{0cm}
\setlength{\cfttabindent}{0cm}

\renewcommand{\figurename}{Abbildung}
\renewcommand{\tablename}{Tabelle}


%%%%%%%%%%%%%%%%%%%%%%%%%%%%%%
%% Definition der Kopfzeile %%
%%%%%%%%%%%%%%%%%%%%%%%%%%%%%%

\pagestyle{fancy}
\renewcommand{\headrulewidth}{0.5pt} %Linie oben
\renewcommand{\footrulewidth}{0.5pt} %Linie oben
\renewcommand{\sectionmark}[1]{\markboth{#1}{}}
\fancyhf{}
\headsep = 1cm
\fancyfoot[R]{\rule[+1.5ex]{0pt}{1ex}\thepage} %Kopfzeile rechts bzw. außen
\fancyfoot[L]{\nouppercase{\leftmark}}

%%%%%%%%%%%%%%%%%%%%%%%%%%%%%%%%%%%%%%%%%%%%%%%%%%%%%
%%  Definition des Deckblattes und der Titelseite  %%
%%%%%%%%%%%%%%%%%%%%%%%%%%%%%%%%%%%%%%%%%%%%%%%%%%%%%

\newcommand{\JMUTitle}[9]{

  \thispagestyle{empty}
  \vspace*{\stretch{1}}
  {\parindent0cm
  \rule{\linewidth}{.7ex}}
  \begin{flushright}
    \vspace*{\stretch{1}}
    \bfseries\Huge
    #1\\
    \vspace*{\stretch{1}}
    \bfseries\large
    #4\\
    \vspace*{\stretch{1}}
    \bfseries\large
    #9
  \end{flushright}
  \rule{\linewidth}{.7ex}

  \vspace*{\stretch{1}}
  \begin{center}
    \includegraphics[width=2in]{siegel} \\
    \vspace*{\stretch{1}}
    \Large #3 \\

    \vspace*{\stretch{2}}
   \large Methoden Business Consulting\\
    \large FOM  M{\"u}nchen\\
    \vspace*{\stretch{1}}
    \large Betreuer:  #8 \\[1mm]
    %\vspace*{\stretch{1}}
    \large Bearbeitungszeitraum: #5 bis #6
    %large W{\"u}rzburg, den #6
  \end{center}
}

\newcommand{\Zusammenfassung}[2]{
    \section*{#1}
    \addcontentsline{toc}{section}{#1}%
    \markboth{#1}{#1}
    \noindent
    #2
    \newpage
}


%%%%%%%%%%%%%%%%%%%%%%%%%%%%
%%  Bibtex File einbinden %%
%%%%%%%%%%%%%%%%%%%%%%%%%%%%



\addbibresource{literatur.bib}  


%%%%%%%%%%%%%%%%%%%%%%%%%%%%
%%  Beginn des Dokuments  %%
%%%%%%%%%%%%%%%%%%%%%%%%%%%%


\begin{document}

\titlespacing{\section}{0pt}{0pt}{0pt}
\titlespacing{\subsection}{0pt}{6pt}{0pt}
\titlespacing{\subsubsection}{0pt}{6pt}{0pt}

  \JMUTitle
      {Irgendwas mit Nachhaltigkeit}                                % Titel der Arbeit
      {Kurztitel der Arbeit}                            % Muss in die Kopfzeile passen
      {Projektarbeit}       % Art der Arbeit
      {Neumair, Marco}                              % Vor- und Nachname des Autors
      {01.03.2023}                                      % Tag der Anemeldung 
      {01.05.2023}                                      % Tag der Abgabe
      {Master Business Consulting & Digital Management}           % Studiengang
      {Prof. Dr. Angela Witt-Bartsch}                       % Name des Betreuers -- Hier sollte *immer* Prof. Winkelmann stehen
      {1234567}                                         % Matrikelnummer 

%%%%%%%%%%%%%%%%%%%%%%%%%%%%
%%  Kurzzusammenfassung / Abstract  %%
%%%%%%%%%%%%%%%%%%%%%%%%%%%%

\newpage
\pagenumbering{Roman}
%%%%%%%%%%%%%%%%%%%%%%%%%%%%%%%%%%%%%%%%%%%%%%

\Zusammenfassung
{Zusammenfassung}
{Die Zusammenfassung dient dem Leser dazu, einen groben Überblick über die Inhalte zu gewinnen (kurze Problemstellung, Herangehensweise, Lösungsansätze und evtl. der Schlüsselerkenntnisse). Der Umfang sollte \underline{ca. eine halbe Seite} betragen. Auf der nächsten Seite soll eine Übersetzung der Zusammenfassung als Abstract in englischer Sprache erfolgen.

Allgemeiner Hinweis: Die „neue“ Rechtschreibung bietet viele alternative Rechtschreibmöglichkeiten. Es ist demnach egal, ob Sie z.B. Potenzial mit „z“ oder Potential mit „t“ schreiben. Auch das Komma kann vor einem erweiterten Infinitiv wahlweise gesetzt oder weggelassen werden. Alternative Schreibweisen bedeuten zugleich aber nicht Beliebigkeit. Sie sollten sich also immer konsequent während der gesamten Arbeit für \underline{eine} Schreibweise entscheiden. Dieses gilt auch für Fachbegriffe.
}

\Zusammenfassung
{Abstract}
{
Zusammenfassung der Seite I in englischer Sprache.

}
%%%%%%%%%%%%%%%%%%%%%%%%%%%%
%%  Inhaltsverzeichnis  %%
%%  wirt automatisch erstellt  %%
%%%%%%%%%%%%%%%%%%%%%%%%%%%%


\tableofcontents


%%%%%%%%%%%%%%%%%%%%%%%%%%%%
%%  Abbildungsverzeichnis %%
%%  wirt automatisch erstellt  %%
%%%%%%%%%%%%%%%%%%%%%%%%%%%%

\newpage

\listoffigures
\addcontentsline{toc}{section}{Abbildungsverzeichnis (ab drei Abb.)}%
\markboth{Abbildungsverzeichnis (ab drei Abb.)}{Abbildungsverzeichnis (ab drei Abb.)}


%%%%%%%%%%%%%%%%%%%%%%%%%%%%
%%  Tabellenverzeichnis   %%
%%  wirt automatisch erstellt  %%
%%%%%%%%%%%%%%%%%%%%%%%%%%%%

\newpage

\addcontentsline{toc}{section}{Tabellenverzeichnis}%
\markboth{Tabellenverzeichnis}{Tabellenverzeichnis}
\listoftables

%%%%%%%%%%%%%%%%%%%%%%%%%%%%
%%  Abkürzungsverzeichnis %%
%%%%%%%%%%%%%%%%%%%%%%%%%%%%

\newpage

\section*{Abkürzungsverzeichnis (nur bei Bachelor- und Masterthesis)}
\addcontentsline{toc}{section}{Abkürzungsverzeichnis}%
\markboth{Abkürzungsverzeichnis}{Abkürzungsverzeichnis}


%%%%%%%%%%%%%%%%%%%%%%%%%%%%
%%  Abkürzungen %%
%%%%%%%%%%%%%%%%%%%%%%%%%%%%
\begin{acronym}[ECU]
\acro{eu}[EU]{Europäische Union}
\end{acronym}



\newpage
\pagenumbering{arabic}
\setcounter{page}{1}
  
  
%%%%%%%%%%%%%%%%%%%%%%%%%%%%
%%  Hauptteil  %%
%%%%%%%%%%%%%%%%%%%%%%%%%%%%


\section{Einleitung} \label{einleitung}
Dieser Teil der Arbeit sollte folgende Inhalte haben:
\begin{itemize}
    \item Einführung in die Problemstellung
    \item Motivation und Herleitung des Themas
    \item Aufbau der Arbeit
\end{itemize}
Grundsätzlich sollten Kapitelüberschriften sprechend sein, das gilt insbesondere für das Einleitungskapitel, denn grundsätzlich ist das erste Kapitel immer ein einleitendes Kapitel. Der Leser würde also mit der Überschrift „Einleitung“ nichts über den Inhalt Ihres Kapitels erfahren. 

Hinweis:

Es hat sich als hilfreich erwiesen, die Einleitung mit der Zusammenfassung bzw. dem Abstract und der Schlussfolgerung zu vergleichen. Damit stellt man sicher, dass diese inhaltlich im Bezug auf Zielsetzung und Motivation übereinstimmen. Der Umfang sollte ca. 5\% der gesamten Arbeit betragen.

\newpage

\section{Kapitelüberschrift} \label{Kapitelüberschrift}
Direkt unterhalb der Hauptkapitel ist jeweils Platz für eine kurze inhaltliche Überleitung.  

\subsection{Gliederung}
Die Überschriftenstrukturierung ist hierarchisch. Es empfiehlt sich, die Arbeit so zu strukturieren, dass der Text immer auf der untersten Ebene steht und zugleich auf gleichen Hierarchieebenen inhaltlich auf gleicher Ebene verfasste Texte stehen. Kapitelüberschriften sind so zu wählen, dass sie sinnvoll den Inhalt des (Unter)Kapitels als Aussage wiedergeben. Die Kapitelüberschriften sind in den Formatvorlagen namentlich als Überschrift 1, Überschrift 2, usw. hinterlegt. \footnote{So geht eine Fußnote. Diese sind jedoch zu vermeiden.}

\subsection{Abbildungen und Tabellen}
Abbildungen werden immer zentriert, mit einer Bezeichnung versehen, eingefügt. 
\begin{figure}[!h]
    \centering
    \includegraphics[width=11cm]{abb1.jpg}
    \caption{Bezeichnung der Abbildung}
    \label{fig:beispielbild}
\end{figure}
\bigskip
\newline
Wichtig ist, Abbildungen immer im Text zu erläutern und im Text auf die Abbildung zu verweisen (vgl. Abbildung \ref{fig:beispielbild}). Dies gilt auch für Tabellen. Bei fremden Abbildungen und Tabellen ist zudem die ursprüngliche Quelle anzugeben (z.B. in Klammern am Ende der Bezeichnung).

Tabellen werden ebenfalls zentriert und mit einer Bezeichnung versehen eingefügt (vgl. Tabelle \ref{Tab1}).

\begin{table}[H]
\centering
\caption{Tabelle}
\label{Tab1}
\def\arraystretch{1.2}%  
\begin{tabular}{ | l | c | r | }
 \hline	
 Name & Type & Description \\ 
  \hline			
  1 & 2 & 3 \\
  \hline  
\end{tabular}
\end{table}

Formatvorlage für Auflistungen und Nummerierungen mit mehreren Ebenen:
\begin{itemize}
    \item Erste Ebene
    \begin{itemize}
        \item Zweite Ebene
        \begin{itemize}
            \item Dritte Ebene
            \begin{itemize}
                \item Vierte Ebene
            \end{itemize}
        \end{itemize}
    \end{itemize}
\end{itemize}


\subsection{Inhalt}
Ihr Text sollte unter Einbindung von Grafiken und Tabellen in Absätze gegliedert werden. Dabei ist zu beachten, dass ein Absatz einen thematischen Gedanken erfasst, wobei am Anfang des Absatzes im Regelfall die Kernaussage zu finden ist und von dieser ausgehend durch weitere Erörterungen innerhalb des Absatzes gegliedert wird. 

\newpage

\section{Einleitung} \label{sec:einleitung}

Eine Untergliederung bis zur dritten Ebene ist sinnvoll. Es emp-fiehlt sich, eine vierte Ebene (z.B. 3.1.1.1) zu vermeiden, um die Übersichtlichkeit der Gliederung zu wahren. Wichtig: sollten Sie unter 3.1 auch noch 3.1.1 gliedern, so ist automatisch auch noch eine Gliederungsunterpunkt 3.1.2 notwendig, denn ein Untergliederpunkt allein ist wenig sinnvoll.

\subsection{Kapitel 3.1}

\subsubsection{Kapitel 3.1.1}

\subsubsection{Beispiele Latex}
\textbf{So wird dick geschrieben} und \textit{so kursiv}. \citet[580]{clemen1989combining} so wird Autor, Jahr und Seite zitiert. So wird in Klammern zitiert: \citep[580]{clemen1989combining} oder (\cites[580]{clemen1989combining}[548]{gilabert2006intelligent}). So wird eine Webquelle zitiert: \citet{shiny1}. So wird referenziert: Kapitel \ref{einleitung}, Gleichung \ref{eq:1} zeigt...

\citet{sun2006information} erklären...

Langfassung Abkürzung: \ac{eu}

Kurzfassung Abkürzung: \acs{eu}

Wie bereits in Kapitel \ref{einleitung} erwähnt.

Gleichung \ref{eq:1} zeigt

\begin{equation}\label{eq:1}
a^2 + b^2 = c^2
\end{equation}

\clearpage

\section{Schlussfolgerung} \label{sec:schlussfolgerung}
In der Schlussfolgerung sollen
\begin{itemize}
    \item die Themenstellung
    \item der gewählte Ansatz
    \item die Ergebnisse der Arbeit
    \item eine kritische Stellungnahme/Einschätzung sowie die Limitationen Ihrer Forschung
    \item nächste Schritte
\end{itemize}
\bigskip
deutlich werden.
\vspace{3mm}
\newline
\textbf{Hinweis}:
Die Schlussfolgerung sollte mit der Zusammenfassung bzw. dem Abstract und der Einleitung abgeglichen werden. Es sollte immer eine Zusammenfassung der wesentlichen Erkenntnisse der eigenen Arbeit sein, die den Forschungsbeitrag darstellt. Der Umfang der Schlussfolgerung sollte ähnlich wie die Einleitung ca. 5\% der gesamten Arbeit betragen.

\newpage
\section*{Literaturverzeichnis} \label{Literaturverzeichnis}
\addcontentsline{toc}{section}{Literaturverzeichnis}%
\markboth{Literaturverzeichnis}{Literaturverzeichnis}

Ein zentrales Kriterium wissenschaftlichen Arbeitens ist die Unterscheidung zwischen eigenen und aus Literaturquellen entnommenen Beiträgen. Zur Kennzeichnung einer Literaturquelle ist die Harvard Citation, die „amerikanische Zitierweise“ oder der Style der MISQ (Link) weit verbreitet. Bei der amerikanischen Zitierweise werden Quellen durch die Nennung des Nachnamens des Autors/der Autorin, das Erscheinungsjahr des Textes sowie die jeweilige(n) Seitenzahl(en), auf die man sich bezieht, direkt im Fließtext angegeben. Zitiert man mehrere Autoren, so werden diese mit einem Semikolon voneinander getrennt. Die vollständigen bibliographischen Informationen werden im Literaturverzeichnis genannt. Es ist zwingend darauf zu achten, dass wirklich jede zitierte Literaturstelle auch im Literaturverzeichnis auftaucht. Es empfiehlt sich für Abschlussarbeiten Literaturverwaltungssoftware, wie beispielsweise Citavi, Mendeley oder EndNote, welche Ihnen kostenlos zur Verfügung stehen, einzusetzen.

\begin{table}[ht]
\centering
\def\arraystretch{1.5}% 
\begin{tabular}{l p{8.5cm}} 
    (Einstein 2001, 24) & Das Zitat bzw. der Verweis bezieht sich auf eine Textstelle auf der Seite 24. \\ 
    (Einstein 2001, 24f.) & Das Zitat bzw. der Verweis bezieht sich auf eine Textstelle, die sich von Seite 24 auf Seite 25 erstreckt.\\ 
    (Einstein 2001, 24ff.) & Der Verweis bezieht sich auf eine Textstelle, die sich von Seite 24 auf Seite 26 erstreckt.\\ 
    (Einstein 2001, 24-29) & Dieser Verweis bezieht sich auf die Seiten 24 bis 29. Diese Form der Seitenangabe verwendet man, wenn man sich auf eine Textstelle bezieht, die sich über mehr als drei Seiten erstreckt.\\ 
    (Einstein 2001, 24 und 27) & Der Verweis bezieht sich auf Textstellen auf den Seiten 24 und 27.\\ 
    (Einstein 2001, 24; Zweistein 2002, 45) & Das Zitat wurde aus zwei unterschiedlichen Werken entnommen.\\ 
    (Einstein und Zweistein 2001, 25) & Das Zitat ist von zwei Autoren. \\ 
    (Einstein et al. 2001, 25) & Das Zitat ist von mehr als zwei Autoren. \\ 

\end{tabular}
\end{table}
\bigskip
\noindent
\textit{Direkte (wörtliche) Zitate}
\vspace{1.5mm}
\newline
Bei einem direkten Zitat muss der zitierte Text originalgetreu wiedergegeben werden, d.h. Rechtschreibfehler oder eine veraltete Orthographie werden unverändert wiedergegeben. Der zitierte Text steht immer in Anführungszeichen. Wird innerhalb eines Zitates ebenfalls zitiert (Zitat im Zitat), so steht das innen stehende Zitat in einfachen Anführungszeichen.
Beispiel: „Das Ergebnis ergab einen Anteil von 73\% für die Nutzung von Open-Source-Software in der öffentlichen Verwaltung“ (Mustermann 2005, 56).
\vspace{3mm}
\newline
\textit{Indirekte (sinngemäße) Zitate}
\vspace{1.5mm}
\newline
Das indirekte Zitat, d.h. die Wiedergabe eines fremden Gedankens mit eigenen Worten, gibt die Meinung eines Autors sinngemäß wieder.
Beispiele:
Nach Ansicht von Mustermann (2005, 56ff.) hat sich die Nutzung von Open-Source-Software in der öffentlichen Verwaltung durchgesetzt.
Aufgrund der Ergebnisse einer Erhebung ist ein eindeutiger Trend erkennbar, dass die öffentliche Verwaltung vermehrt auf den Einsatz von Open-Source-Software setzt (Mustermann 2005, 56ff.).
\vspace{3mm}
\newline
\textit{Erlaubte Änderungen in Zitaten sind:}
\vspace{1.5mm}
\newline
Auslassungen (Ellipsen)
\newline
Auslassungen in einem direkten Zitat sind erlaubt, wenn der Sinn der ursprünglichen 
\newline
Belegstelle nicht verstellt wird. Auslassungen werden durch eine eckige Klammer mit drei Punkten [...] gekennzeichnet. 
\newline
Beispiel: Das Ergebnis einer Erhebung ergab einen „[…] Anteil von 73\% für die Nutzung von Open-Source-Software in der öffentlichen Verwaltung“ (Mustermann 2005, 56).
\vspace{3mm}
\newline
\textit{Ergänzungen (Interpolation)}
In das Originalzitat eingefügte Wörter zur grammatikalischen Angleichung werden in
eine eckige Klammer [ ] an die passende Stelle gesetzt.
Beispiel:
In einer ähnlichen Studie wurde ein „[…] Anteil von 73\% für die Nutzung von Open-Source-Software in der öffentlichen Verwaltung [ermittelt]“ (Mustermann 2005, 56).
\vspace{3mm}
\newline
\newpage
\noindent
\textit{Literaturverzeichnis}
\vspace{1.5mm}
\newline
Zu jeder wissenschaftlichen Arbeit gehört ein Literaturverzeichnis, in dem alle zitierten Quellen in alphabetischer Reihenfolge enthalten sind. Die Literaturangaben werden alphabetisch nach Zuname des Autors, dann chronologisch geordnet. Aufgrund der Übersichtlichkeit soll im Literaturverzeichnis ab der 2ten Zeile einer Quelle der Text eingerückt werden.
\vspace{3mm}
\newline
Es gibt folgende unterschiedliche Literaturquellen, die jedoch im Literaturverzeichnis alle in eine Liste geschrieben werden:
\vspace{3mm}
\newline
\textbf{Bücher/Monographien} werden mit Verfasser bzw. Herausgeber, Erscheinungsjahr, Titel und Untertitel (kursiv), Auflage, Verlag und Erscheinungsort angegeben.
\begin{itemize}
    \item Name, Vorname[; Name, Vorname] (Erscheinungsjahr): Titel. Ggf. Untertitel. Auflage (wenn nicht Erstauflage), Verlag, Erscheinungsort(e).
    \item Ein Autor: Burchardt, M. (1996): Leichter studieren: Wegweiser für effektives wissenschaftliches Arbeiten. 2. Aufl., Berliner Wissenschafts-Verlag, Berlin.
    \item Zwei Autoren: Tribowski, C.; Tamm, G. (2010): RFID: Informatik im Fokus. Springer Verlag, Berlin.
    \item Drei oder mehr Autoren: Schuster, E. W.; Allen, S. J.; Brock, D. L. (2006): Global RFID – The Value of the EPCglobal Network for Supply Chain Management. Springer Verlag, Berlin.
    \item Mehr als eine Quelle vom selben Autor und im selben Jahr: \\ Müller, A. (2008a): BANF anlegen für Anfänger – Eine empirische Studie. Heindle Verlag, Berlin. \\
    Müller, A. (2008b): BANF anlegen für Fortgeschrittene. 3. Aufl., Springer Verlag, Berlin.
\end{itemize}
\bigskip
\newpage
\noindent
\textbf{Zeitschriftenaufsätze} erfordern folgende Angaben:
\begin{itemize}
    \item Name, Vorname[; Name, Vorname] (Erscheinungsjahr): Titel des Beitrags (kursiv), Namen und Jahrgang der Zeitschrift, Jahrgangsnummer (Heftnummer), Seitenzahl.
    \item In der Regel umfasst ein Jahrgang mehrere Einzelhefte. Dabei werden die Seitenzahlen durchgezählt, d.h. dass z.B. Heft 1 Seite 1-234 umfasst, Heft 2 Seite 235-456, Heft 3 Seite 457-640 und Heft 4 Seite 641-865. Manche Zeitschriften zählen jedoch jedes Heft eines Jahrgangs einzeln, so dass jedes Heft wieder bei Seite 1 beginnt. In diesem Fall müssen Sie hinter der Jahrgangsnummer auch die Heftnummer in Klammern angeben.
    \item Borchert, M. (1983): Einige außenwirtschaftliche Aspekte staatlicher Verschuldung. In: Kredit und Kapital, 16, 513-527.
    \item Endeward, D.; Stettner, P. (2000): Zitieren von elektronischen Dokumenten. In: Computer + Unterricht, 10 (40), 37-39.
    \item Wenn Sie Abkürzungen für Zeitschriften verwenden möchten, z.B. ISR für „Information Systems Research“, so integrieren Sie diese in das Abkürzungsverzeichnis.
\end{itemize}
\bigskip
Beiträge aus \textbf{Sammelbänden} werden mit dem Namen des Verfassers, dem Erscheinungsjahr in Klammern und dem Titel des Beitrags (kursiv) erfasst sowie mit dem Namen des Herausgebers des Sammelbandes, dem Titel des Sammelbandes, dem Verlag, dem Verlagsort und der Seitenzahl:
\begin{itemize}
    \item Bender, D. (1983): Nettoinvestition, Lohnbildung und Beschäftigung bei flexiblen Wechselkursen. In: Feldsieper, M.; Groß, R. (Hrsg.): Wirtschaftspolitik in weltoffener Wirtschaft. Duncker und Humblot, Berlin, 29-45. 
\end{itemize}
\bigskip
\newpage
\noindent
Bei \textbf{Internetquellen} sind der Autor, der Titel der Veröffentlichung (kursiv), die Internetadresse und das Datum des Zugriffs anzugeben.
\begin{itemize}
    \item Krugman, P.R. (2014): Currency Crises. In: http://web.mit.edu/crises.html, zugegriffen am 01.06.2014.
\end{itemize}

\newpage


\DeclareDelimFormat[bib,biblist]{nametitledelim}{\addcolon\space}
\DeclareFieldFormat
  [article,inbook,incollection,inproceedings,patent,thesis,unpublished]
  {title}{\mkbibemph{#1}\isdot}
\DeclareFieldFormat
  [article,inbook,incollection,inproceedings,patent,thesis,unpublished]
  {journaltitle}{#1\isdot}
\DeclareFieldFormat{pages}{#1}
\DeclareFieldFormat
  [article,inbook,incollection,inproceedings,patent,thesis,unpublished]
  {authortitle}{#1\isdot}

\printbibliography[title= Literaturverzeichnis]

\newpage
\section*{Anhang} \label{Anhang}
\addcontentsline{toc}{section}{Anhang}%
\markboth{Anhang}{Anhang}
Ein Anhang zur wissenschaftlichen Arbeit ist notwendig, wenn Materialien, die die Arbeit als Ganzes oder auch größere Teile derselben betreffen, jedoch nur schwer im Ausführungsteil unterzubringen sind. Das ist insbesondere dann der Fall, wenn sie aufgrund ihres Umfangs den Gesamtzusammenhang der Ausführung stören würden. Inhaltlich darf im Anhang nichts stehen, was zum Verständnis des Textes notwendig ist, der Text der Arbeit darf an dieser Stelle nicht „unter anderen Vorzeichen“ fortgesetzt werden. Er sollte nicht dazu verwendet werden, der Arbeit einen größeren Umfang zu geben und diese „dicker“ erscheinen zu lassen!
\vspace{3mm}
\newline
Der Anhang eignet sich für ergänzende Dokumente und Materialien, vor allem, falls diese für den Leser nur schwer oder gar nicht zugänglich sind, wie bspw. unveröffentlichte Betriebsunterlagen.
\vspace{3mm}
\newline
Vor allem in den empirischen Arbeiten kann der Anhang dazu dienen, verwendete Datensätze, eingesetzte mathematischstatistische Verfahren oder Programme näher zu kennzeichnen. Werden im Rahmen der Untersuchungen Befragungen durchgeführt, sind die Fragestellungen und Ergebnisse im Anhang zu dokumentieren. Auf Gespräche darf im Rahmen der Ausführungen nur dann Bezug genommen werden, wenn ein vom Gesprächspartner unterzeichnetes Ergebnis-Protokoll im Anhang der Arbeit beigefügt ist.
\vspace{3mm}
\newline
Besteht der Anhang aus mehreren Elementen, so sind die einzelnen Elemente durch Nummerierung voneinander zu trennen. In diesem Falle ist ein entsprechendes Anhangsverzeichnis zu erstellen.
\vspace{3mm}
\newline
(Bleibt der Anhang leer, kann dieser Abschnitt gelöscht werden)
\newpage

\input{Erklaerung.tex}

\end{document}